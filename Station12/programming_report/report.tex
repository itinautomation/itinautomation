\documentclass[12pt]{article}

\begin{document}
\title{Programming Report for Stations 1 and 2}
\author{
    Christoph Echtinger-Sieghart \textbf{(00304130)} \\
    Thomas Hartberger \textbf{(01327876)}
}
\maketitle

\begin{abstract}
After a brief overview of the task at hand, we will discuss the methodology used for implementing
the required functionality. First we discuss our high-level approach and then provide some
implementation details. We finish this report with a quick overview of the Human-Machine-Interface.
\end{abstract}

\section{Overview}
We were tasked to program the Stations 1 (``Vereinzeln'') and 2 (``Pr\"ufen'') according to
the specification derived during previous modelling using SysML. The programming was to be
done using the Siemens TIA Portal application. Since port assignments of the various sensors
and actuators were not know beforehand, the first task was to derive these assignments
and to check if all the sensors and actuators were working as specified. During this process
we discovered that the sensor for detecting the end position of the magazine's slider was
non-operational. Since this sensor's data is not critical, we decided to go ahead and create a virtual sensor
using a timer based solution. 

\section{Implementation}

\subsection{High-level Implementation}

Stations 1 and 2 naturally decompose into three finite-state machines. The first state machine being
Station 1, the second state machine being Station 2 without the ramp and the third state machine
represents the ramp. These three state machines run in parallel, but some state transitions are dependent
on another state machine being in a specific state. These dependencies synchronize the state machines
at critical points of the operation.
A special state machine called \texttt{Initialization} is responsible for getting the system into
a well-defined state. The \texttt{Initialization} state machine is always executed before the system
starts normal operation.

A detailed specification of the state machines and their interaction can be found in the
combined SysML model created for Stations 1 and 2.

\subsection{Detailed Implementation}

The major part of the implementation is done using \textbf{FBD} - for the small remainder we
used \textbf{SCL}.
We used a structured approach during implementation. First we encapsulated all of the actuators
into functional blocks that provide a uniform interface (see Table~\ref{table:interface}).

\begin{table}[h]
\begin{minipage}[b][1.5cm][t]{0.4\textwidth}
    \centering
    \begin{tabular}{|l|}
    Inputs \\
    \hline
    \texttt{enable}  \\
    \texttt{move\_*} \\
    \end{tabular}%
\end{minipage}%
\begin{minipage}[b][1.5cm][t]{0.4\textwidth}%
    \centering
    \begin{tabular}{|l|}
    Outputs \\
    \hline
     \texttt{position\_reached} \\
    \end{tabular}
\end{minipage}%
\caption{\label{table:interface} Interface for actuator blocks}
\end{table}

The general idea of the interface is as follows: Whenever an actuator is allowed to move it's \texttt{enable}
input is high. Since all our actuators know only of two directions in which to move, the \texttt{move\_*} input
is enough to specify the desired direction of movement. Whenever an actuator has finished moving in the
desired direction the output \texttt{position\_reached} goes high.

This interface for the actuators was then combined with the state machine approach. The \texttt{enable}
inputs of the actuators function as a kind of sensitivity list with regard to the states where they should be
active. The state transitions are then often dependent on the output \texttt{position\_reached} going high and
some additional constraints. In the case of an emergency shutdown a special state that does not enable any actuators is entered.

\section{Human-Machine-Interface}

To start normal operation the operator must press the green button. A controlled shutdown
can be initiated by pressing the yellow button. Passing a material unit on to the next station
is done by pressing the white button. Pressing the red button initiates an emergency shutdown and
halts operation immediately.

A detailed description of the Human-Machine-Interface can be found in the
use-case diagram in the combined SysML model created for Stations 1 and 2.
\end{document}
